% Formato de Informe para Sistemas de Información I
% Universidad Católica del Norte - Coquimbo
% Primer Semestre 2025

\documentclass[12pt,letterpaper]{report}

% Paquetes necesarios
\usepackage[spanish]{babel}
\usepackage[utf8]{inputenc}
\usepackage{graphicx}
\usepackage{fancyhdr}
\usepackage{hyperref}
\usepackage{booktabs}
\usepackage{tabularx}
\usepackage{xcolor}
\usepackage{enumitem}
\usepackage{titlesec}
\usepackage{float}

% Configuración de márgenes
\usepackage[top=2.5cm, bottom=2.5cm, left=3cm, right=3cm]{geometry}

% Configuración de títulos
\titleformat{\chapter}{\normalfont\LARGE\bfseries}{\thechapter.}{1em}{}
\titleformat{\section}{\normalfont\Large\bfseries}{\thesection}{1em}{}
\titleformat{\subsection}{\normalfont\large\bfseries}{\thesubsection}{1em}{}

% Configuración de encabezado y pie de página
\pagestyle{fancy}
\fancyhf{}
\fancyhead[L]{Sistemas de Información I}
\fancyhead[R]{Universidad Católica del Norte}
\fancyfoot[C]{\thepage}
\renewcommand{\headrulewidth}{0.4pt}
\renewcommand{\footrulewidth}{0.4pt}

% Configuración de hipervínculos
\hypersetup{
    colorlinks=true,
    linkcolor=blue,
    filecolor=magenta,      
    urlcolor=cyan,
    pdftitle={Proyecto de Sistemas de Información},
    pdfpagemode=FullScreen,
}

\begin{document}

%------------------------------------------------
% PORTADA
%------------------------------------------------
\begin{titlepage}
    \centering
    \vspace*{1cm}
    \includegraphics[width=0.4\textwidth]{ucn-EIC.pdf}
    \vspace{1cm}
    
    {\LARGE \textbf{UNIVERSIDAD CATÓLICA DEL NORTE}}\\
    \vspace{0.5cm}
    {\large Escuela de Ingeniería}\\
    \vspace{0.3cm}
    {\large ICCI - ITI}\\
    \vspace{0.3cm}
    {\large Coquimbo}\\
    \vspace{1.5cm}
    
    {\Huge \textbf{Proyecto de Sistemas de Información}}\\
    \vspace{0.5cm}
    {\LARGE Sistemas de Información I}\\
    \vspace{1.5cm}
    
    {\large \textbf{Integrantes:}}\\
    \vspace{0.3cm}
    {\large Daniela Castro }
    {\large Vicente Espinoza }\\
    {\large Diego Martínez}\\
    {\large Sebastián Vega}\\
    {\large Gabriel Vergara}\\
    \vspace{1.5cm}
    
    {\large \textbf{Profesor:} Felipe Quiroz}\\
    \vspace{1.5cm}
    
    {\large Primer Semestre 2025}
    
\end{titlepage}

%------------------------------------------------
% ÍNDICE
%------------------------------------------------
\tableofcontents
\newpage

%------------------------------------------------
% INTRODUCCIÓN
%------------------------------------------------
\chapter*{Introducción}
\addcontentsline{toc}{chapter}{Introducción}

La formacion de profesionales es uno de los pilares principales para las universidades, para ello deben pasar un proceso riguroso de varios años para poder salir a un campo laboral a poder aplicar lo aprendido. Sin embargo, no todas las personas que entran a universiades y/o insitutos tienden a completar este proceso de educacion superior, donde el rendimiento de cada estudiante corresponde a una de las cuantas razones validas para que ocurra este suceso. Es por eso que la Universidad Catolica del Norte, comprometida con la excelencia educativa y academica, ha identificado la necesidad de analizar habitos y estilos de vida de los estudiantes de su institucion para determinar como estos impactan en su desempeño universitario.

Durante este proyecto, se busca abordar esta problematica mediante el análisis de un dataset simulado que contiene el registro de 1.000 estudiantes, con mas de 15 variables a analizar las cuales seran detalladas mas adelante. Para ello, se utilizaran tecnicas de analisis de datos, visualizacion en Power BI y KPIs relevantes. Esto permitira tomar decisiones informadas para impulsar y mejorar el desempeño estudiantil.



\newpage

%------------------------------------------------
% DEFINICIÓN "ORGANIZACIÓN E INDUSTRIA"
%------------------------------------------------
\chapter{Definición de la Organización e Industria}

\section{Análisis de la Organización}

\subsection{Misión}
La Universidad Católica del Norte inspirada en los principios del Humanismo Cristiano y la misión de la Iglesia Católica, contribuye a la creación y transferencia del conocimiento, a la formación integral de la persona y el desarrollo tecnológico. Como institución con vocación de servicio y excelencia impulsa desde el Norte de Chile, con las comunidades y el territorio, la sostenibilidad a través de la docencia, investigación y vinculación con el medio

\subsection{Visión}
Desde su identidad católica y vocación de excelencia, ser una universidad referente en su quehacer, que inspirada en el bien común integre disciplinas, tradiciones, culturas y comunidades para transformar vidas y ampliar oportunidades. 

\subsection{Descripción Detallada}
La Universidad Católica del Norte (UCN) fundada en el año 1956, es una institucion privada de educacion superior que reside en Chile, especificamente en las ciudades de Coquimbo y Antofagasta, siendo esta ultima ciudad donde se encuentra la sede principal. Esta universidad forma parte del Consejo de Rectores de las Universidades Chilenas (CRUCH) y de la Red G9, que agrupa a las principales universidades tradicionales no estatales del país. Cuenta  actualmente con mas de 50 carreras de pregrado, las cuales en su mayoria se encuentran en su sede principal, pero tambien ofrece programas de postgrado,magisteres, doctorados y diplomados

\section{Identificación de Stakeholders}
[Identificar y describir los principales grupos de interés (stakeholders) relacionados con la organización y el proyecto]

\begin{table}[H]
    \centering
    \begin{tabularx}{\textwidth}{|X|X|X|}
        \hline
        \textbf{Stakeholder} & \textbf{Rol/Relación} & \textbf{Intereses/Expectativas} \\
        \hline
        & & \\
        \hline
        & & \\
        \hline
        & & \\
        \hline
    \end{tabularx}
    \caption{Identificación de Stakeholders}
\end{table}

\newpage

%------------------------------------------------
% DEFINICIÓN "DATA SET"
%------------------------------------------------
\chapter{Definición del Data Set}

\section{Descripción del Data Set}
[Describir el conjunto de datos que se utilizará en el proyecto, incluyendo su origen, formato, estructura, variables principales, etc.]

\section{Justificación del Data Set}
[Explicar por qué se eligió este conjunto de datos específico y cómo se relaciona con el problema a resolver]

\section{Calidad y Limitaciones de los Datos}
[Analizar la calidad de los datos, identificar posibles problemas (datos faltantes, inconsistencias, etc.) y limitaciones del conjunto de datos]

\newpage

%------------------------------------------------
% PASOS DEL PROYECTO
%------------------------------------------------
\chapter{Desarrollo del Proyecto}

%------------------------------------------------
% PASO 1: ANÁLISIS DE LA ORGANIZACIÓN
%------------------------------------------------
\section{Paso 1: Análisis de la Organización}
\textit{Nota: Algunas de estas informaciones ya fueron presentadas en el capítulo de "Definición de la Organización e Industria". Se puede hacer referencia a dicho capítulo y complementar con información adicional si es necesario.}

\subsection{Factores Organizacionales Centrales}
[Analizar los factores organizacionales centrales mencionados en el documento:
\begin{itemize}
    \item El entorno en que debe funcionar la organización
    \item La estructura de la organización: jerarquía, especialización, rutinas y procesos de negocios
    \item La cultura y las políticas de la organización
    \item El tipo de organización y su estilo de liderazgo
    \item Los principales grupos de interés afectados por el sistema
    \item Los tipos de tareas, decisiones y procesos de negocios que el sistema de información ayudará
\end{itemize}]

%------------------------------------------------
% PASO 2: INTRODUCCIÓN A LA PROBLEMÁTICA
%------------------------------------------------
\section{Paso 2: Introducción a la Problemática}

\subsection{Descripción del Problema}
[Describir detalladamente el problema o la oportunidad de mejora identificada en la organización]

\subsection{Justificación}
[Explicar por qué es importante resolver este problema o implementar esta mejora]

\subsection{Impacto Esperado}
[Describir el impacto esperado de la solución propuesta en la organización]

%------------------------------------------------
% PASO 3: IDENTIFICACIÓN Y DETALLE DEL PROCESO
%------------------------------------------------
\section{Paso 3: Identificación y Detalle del Proceso}

\subsection{Descripción General del Proceso}
[Describir el proceso sobre el cual se va a trabajar]

\subsection{Diagrama del Proceso}
[Incluir un diagrama que represente visualmente el proceso]

\subsection{Detalle de Cada Paso del Proceso}
[Explicar cada uno de los pasos del proceso identificado]

\begin{table}[H]
    \centering
    \begin{tabularx}{\textwidth}{|X|X|X|X|}
        \hline
        \textbf{Paso} & \textbf{Descripción} & \textbf{Responsable} & \textbf{Recursos necesarios} \\
        \hline
        & & & \\
        \hline
        & & & \\
        \hline
        & & & \\
        \hline
    \end{tabularx}
    \caption{Detalle de los Pasos del Proceso}
\end{table}

%------------------------------------------------
% PASO 4: IDENTIFICACIÓN DE PROBLEMA O MEJORA
%------------------------------------------------
\section{Paso 4: Identificación de Problema o Mejora}

\subsection{Árbol de Problemas}

El árbol de problemas es una herramienta de análisis que permite identificar y visualizar de manera estructurada un problema central, sus causas y sus efectos. Esta metodología facilita la comprensión integral de la problemática al establecer relaciones causales entre diferentes factores, permitiendo así un enfoque más efectivo para el diseño de soluciones.

En el contexto del rendimiento académico estudiantil de la Universidad Católica del Norte, se ha identificado como problema central: \textbf{"Rendimiento académico insuficiente en estudiantes de la Universidad Católica del Norte"}.

\subsubsection{Causas del Problema}

\textbf{Causas Directas:}
\begin{itemize}
    \item Hábitos de estudio inadecuados.
    \item Baja asistencia a clases.
    \item Deficiente gestión del tiempo.
    \item Falta de apoyo académico personalizado.
\end{itemize}

\textbf{Causas Indirectas:}

\textbf{\textit{Factores de Estilo de Vida:}}
\vspace{0.1cm}
\begin{itemize}
    \item Excesivo tiempo en redes sociales y entretenimiento digital.
    \item Horarios de sueño irregulares o insuficientes.
    \item Mala calidad de alimentación.
    \item Sedentarismo y falta de ejercicio.
\end{itemize}

\textbf{\textit{Factores Socioeconómicos:}}
\vspace{0.1cm}
\begin{itemize}
    \item Necesidad de trabajar medio tiempo.
    \item Limitado acceso a tecnología de calidad.
    \item Bajo nivel educativo de los padres.
    \item Estrés financiero familiar.
\end{itemize}

\textbf{\textit{Factores Institucionales:}}
\vspace{0.1cm}
\begin{itemize}
    \item Sistemas de seguimiento académico limitados.
    \item Falta de identificación temprana de estudiantes en riesgo.
    \item Insuficiente orientación en técnicas de estudio.
\end{itemize}

\subsubsection{Efectos del Problema}

\textbf{Efectos Directos:}
\vspace{0.1cm}
\begin{itemize}
    \item Bajas calificaciones en exámenes.
    \item Alto índice de reprobación de asignaturas.
    \item Deterioro de la salud mental estudiantil.
    \item Desmotivación académica.
\end{itemize}

\textbf{Efectos Indirectos:}
\vspace{0.1cm}
\begin{itemize}
    \item Aumento de la deserción universitaria.
    \item Reducción de la empleabilidad de egresados.
    \item Deterioro del prestigio institucional.
    \item Pérdida de recursos económicos.
\end{itemize}

\subsection{Matriz de Vester}

La Matriz de Vester es una herramienta de análisis que permite evaluar la influencia que ejercen los diferentes problemas identificados entre sí, facilitando su priorización según su grado de motricidad y dependencia. Esta metodología clasifica los problemas en cuatro categorías: críticos, activos, pasivos e indiferentes, permitiendo enfocar los esfuerzos en aquellos que generarán mayor impacto.

Para el análisis del rendimiento académico estudiantil, se han identificado los siguientes problemas:

\begin{itemize}
    \item \textbf{P1}: Hábitos de estudio inadecuados.
    \item \textbf{P2}: Excesivo uso de redes sociales y entretenimiento.
    \item \textbf{P3}: Horarios de sueño irregulares.
    \item \textbf{P4}: Falta de ejercicio físico.
    \item \textbf{P5}: Necesidad de trabajar medio tiempo.
    \item \textbf{P6}: Limitado acceso a tecnología.
    \item \textbf{P7}: Bajo apoyo familiar en educación.
    \item \textbf{P8}: Sistema de seguimiento académico limitado.
\end{itemize}

\begin{table}[H]
    \centering
    \begin{tabular}{|c|c|c|c|c|c|c|c|c|c|}
        \hline
        \textbf{Problemas} & \textbf{P1} & \textbf{P2} & \textbf{P3} & \textbf{P4} & \textbf{P5} & \textbf{P6} & \textbf{P7} & \textbf{P8} & \textbf{Total Activos} \\
        \hline
        \textbf{P1} & 0 & 2 & 3 & 1 & 1 & 1 & 0 & 0 & \textbf{8} \\
        \hline
        \textbf{P2} & 3 & 0 & 3 & 2 & 0 & 0 & 0 & 0 & \textbf{8} \\
        \hline
        \textbf{P3} & 3 & 1 & 0 & 2 & 1 & 0 & 0 & 0 & \textbf{7} \\
        \hline
        \textbf{P4} & 2 & 0 & 1 & 0 & 0 & 0 & 0 & 0 & \textbf{3} \\
        \hline
        \textbf{P5} & 2 & 1 & 2 & 3 & 0 & 0 & 1 & 0 & \textbf{9} \\
        \hline
        \textbf{P6} & 1 & 0 & 0 & 0 & 0 & 0 & 0 & 1 & \textbf{2} \\
        \hline
        \textbf{P7} & 2 & 0 & 1 & 1 & 2 & 3 & 0 & 0 & \textbf{9} \\
        \hline
        \textbf{P8} & 1 & 0 & 0 & 0 & 0 & 0 & 0 & 0 & \textbf{1} \\
        \hline
        \textbf{Total Pasivos} & \textbf{14} & \textbf{4} & \textbf{10} & \textbf{9} & \textbf{4} & \textbf{4} & \textbf{1} & \textbf{1} & \\
        \hline
    \end{tabular}
    \caption{Matriz de Vester - Influencia entre Problemas del Rendimiento Académico}
\end{table}

\subsubsection{Clasificación de Problemas según Matriz de Vester}

\textbf{Problemas Críticos (Alto Activo, Alto Pasivo):}
\begin{itemize}
    \item P1: Hábitos de estudio inadecuados (8,14).
    \item P3: Horarios de sueño irregulares (7,10).
\end{itemize}

\textbf{Problemas Activos (Alto Activo, Bajo Pasivo):}
\begin{itemize}
    \item P5: Necesidad de trabajar medio tiempo (9,4).
    \item P7: Bajo apoyo familiar en educación (9,1).
\end{itemize}

\textbf{Problemas Pasivos (Bajo Activo, Alto Pasivo):}
\begin{itemize}
    \item P4: Falta de ejercicio físico (3,9).
\end{itemize}

\textbf{Problemas Indiferentes (Bajo Activo, Bajo Pasivo):}
\begin{itemize}
    \item P2: Excesivo uso de redes sociales (8,4).
    \item P6: Limitado acceso a tecnología (2,4).
    \item P8: Sistema de seguimiento limitado (1,1).
\end{itemize}

\subsection{Cadena de Valor}

La cadena de valor es un modelo que describe las actividades de una organización para generar valor al cliente final. Desarrollada por Michael Porter, esta herramienta permite identificar las actividades primarias y de apoyo que contribuyen a la ventaja competitiva. En el contexto universitario, la cadena de valor nos ayuda a entender cómo el proceso de rendimiento académico estudiantil se integra con las diferentes actividades institucionales y dónde se pueden implementar mejoras.

\subsubsection{Actividades Primarias de la UCN}

\textbf{Docencia (Enseñanza-Aprendizaje):}
\begin{itemize}
    \item \textbf{\textit{Ubicación del proceso}:} El rendimiento académico es el resultado directo de esta actividad.
    \item \textbf{\textit{Impacto}:} Directamente afectado por la calidad de los procesos de seguimiento y apoyo estudiantil.
    \item \textbf{\textit{Oportunidad de mejora}:} Implementación de sistemas de monitoreo continuo del progreso académico.
\end{itemize}

\textbf{Investigación:}
\begin{itemize}
    \item \textbf{\textit{Relación}:} Los estudiantes con mejor rendimiento pueden participar más activamente en proyectos de investigación.
    \item \textbf{\textit{Impacto}:} El bajo rendimiento limita la participación estudiantil en actividades de investigación.
    \item \textbf{\textit{Oportunidad}:} Usar la investigación como herramienta motivacional para mejorar el rendimiento.
\end{itemize}

\textbf{Vinculación con el Medio:}
\begin{itemize}
    \item \textbf{\textit{Relación}:} El rendimiento académico afecta la calidad de los profesionales que se vinculan con el entorno.
    \item \textbf{\textit{Impacto}:} Estudiantes con mejor rendimiento representan mejor a la universidad en actividades externas.
    \item \textbf{\textit{Oportunidad}:} Establecer programas de vinculación que motiven el mejor rendimiento académico.
\end{itemize}

\subsubsection{Actividades de Apoyo de la UCN}

\textbf{Gestión de Recursos Humanos:}
\begin{itemize}
    \item \textbf{\textit{Relación}:} Formación docente en técnicas de seguimiento académico y detección temprana de problemas.
    \item \textbf{\textit{Oportunidad}:} Capacitación especializada en metodologías de apoyo estudiantil personalizado.
\end{itemize}

\textbf{Desarrollo Tecnológico:}
\begin{itemize}
    \item \textbf{\textit{Relación}:} Sistemas de información para seguimiento académico y análisis predictivo.
    \item \textbf{\textit{Oportunidad}:} Implementación de analytics educativo para predecir y prevenir bajo rendimiento.
\end{itemize}

\textbf{Infraestructura:}
\begin{itemize}
    \item \textbf{\textit{Relación}:} Espacios de estudio, bibliotecas, laboratorios y tecnología disponible.
    \item \textbf{\textit{Impacto}:} La calidad de infraestructura afecta directamente las condiciones de estudio.
\end{itemize}

\textbf{Adquisiciones:}
\begin{itemize}
    \item \textbf{\textit{Relación}:} Recursos tecnológicos, bibliográficos y herramientas de apoyo académico.
    \item \textbf{\textit{Oportunidad}:} Inversión estratégica en herramientas de seguimiento y apoyo estudiantil.
\end{itemize}

\subsection{Análisis FODA}

El análisis FODA es una herramienta de diagnóstico estratégico que permite evaluar la situación interna y externa de una organización. En el contexto del rendimiento académico estudiantil, el FODA nos ayuda a identificar los elementos internos y externos que influyen en esta problemática y a desarrollar estrategias específicas para su mejora.

\begin{table}[H]
    \centering
    \begin{tabular}{|p{0.45\textwidth}|p{0.45\textwidth}|}
        \hline
        \textbf{Fortalezas} & \textbf{Oportunidades} \\
        \hline
        \begin{itemize}
            \item Prestigio y trayectoria de 60+ años en educación superior.
            \item Enfoque humanista cristiano que promueve formación integral.
            \item Ubicación estratégica en el norte de Chile.
            \item Cuerpo académico calificado con experiencia.
            \item Infraestructura tecnológica en desarrollo.
            \item Programas acreditados y reconocidos.
            \item Cultura organizacional orientada a las personas.
        \end{itemize} & 
        \begin{itemize}
            \item Creciente demanda de educación superior en la región.
            \item Avances tecnológicos en analytics educativo y big data.
            \item Políticas gubernamentales de apoyo a la educación superior.
            \item Alianzas estratégicas con organizaciones públicas y privadas.
            \item Desarrollo de metodologías de enseñanza innovadoras.
            \item Programas de bienestar estudiantil en expansión.
            \item Tendencia hacia la personalización educativa.
        \end{itemize} \\
        \hline
        \textbf{Debilidades} & \textbf{Amenazas} \\
        \hline
        \begin{itemize}
            \item Limitado sistema de seguimiento individual del rendimiento académico.
            \item Falta de identificación temprana de estudiantes en riesgo.
            \item Limitada integración de datos estudiantiles para análisis predictivo.
            \item Recursos limitados para programas de bienestar estudiantil.
            \item Falta de capacitación docente en seguimiento académico.
            \item Sistemas de información académica no integrados.
        \end{itemize} & 
        \begin{itemize}
            \item Competencia creciente de otras universidades en la región.
            \item Cambios socioeconómicos que afectan el acceso a educación superior.
            \item Impacto de factores externos en el bienestar estudiantil.
            \item Limitaciones presupuestarias del sector educativo.
            \item Rápidos cambios tecnológicos que requieren constante actualización.
            \item Expectativas crecientes de estudiantes y empleadores.
            \item Factores socioeconómicos familiares que afectan el rendimiento.
        \end{itemize} \\
        \hline
    \end{tabular}
    \caption{Análisis FODA - Rendimiento Académico Estudiantil UCN}
\end{table}

\subsubsection{Estrategias Derivadas del Análisis FODA}

\textbf{Estrategias FO (Fortalezas + Oportunidades):}
\begin{itemize}
    \item Aprovechar el prestigio institucional y los avances tecnológicos para desarrollar un sistema líder en seguimiento académico en la región.
    \item Utilizar la cultura orientada a personas y las metodologías innovadoras para personalizar el apoyo estudiantil.
    \item Capitalizar la ubicación estratégica y las alianzas para crear programas de apoyo integral únicos en el norte de Chile.
\end{itemize}

\textbf{Estrategias FA (Fortalezas + Amenazas):}
\begin{itemize}
    \item Usar la trayectoria institucional y programas acreditados para diferenciarse de la competencia creciente.
    \item Aprovechar el enfoque humanista para contrarrestar factores socioeconómicos adversos que afectan a los estudiantes.
    \item Utilizar el cuerpo académico calificado para adaptarse rápidamente a los cambios tecnológicos.
\end{itemize}

\textbf{Estrategias DO (Debilidades + Oportunidades):}
\begin{itemize}
    \item Desarrollar sistemas integrados de seguimiento académico aprovechando los avances en analytics educativo.
    \item Implementar programas de apoyo personalizado usando las políticas gubernamentales de apoyo disponibles.
    \item Capacitar al personal docente en metodologías innovadoras para la identificación temprana de problemas académicos.
\end{itemize}

\textbf{Estrategias DA (Debilidades + Amenazas):}
\begin{itemize}
    \item Fortalecer los sistemas de información académica para responder mejor a las expectativas crecientes de estudiantes y empleadores.
    \item Mejorar la capacitación docente y los programas de bienestar para enfrentar los cambios tecnológicos y socioeconómicos.
    \item Desarrollar alianzas estratégicas para compensar las limitaciones presupuestarias y de recursos.
\end{itemize}
%------------------------------------------------
% PASO 5: DEFINICIÓN DE OBJETIVOS, ALCANCE Y SOLUCIÓN
%------------------------------------------------
\section{Paso 5: Definición de Objetivos, Alcance y Solución}

\subsection{Objetivos}

\subsubsection{Objetivo General}
[Definir el objetivo general del proyecto]

\subsubsection{Objetivos Específicos}
[Listar y describir los objetivos específicos del proyecto]

\subsection{Alcance}
[Definir el alcance del proyecto, especificando lo que se incluirá y lo que quedará fuera]

\subsection{Descripción de la Solución Propuesta}
[Describir detalladamente la solución propuesta para resolver el problema identificado]

%------------------------------------------------
% PASO 6: PLANIFICACIÓN
%------------------------------------------------
\section{Paso 6: Planificación}

\subsection{Ciclo de Desarrollo}
[Describir el ciclo de desarrollo que se utilizará para implementar la solución]

\subsection{División del Trabajo y Roles del Equipo}
[Especificar cómo se dividirá el trabajo y qué rol desempeñará cada miembro del equipo]

\begin{table}[H]
    \centering
    \begin{tabularx}{\textwidth}{|X|X|X|}
        \hline
        \textbf{Miembro del Equipo} & \textbf{Rol} & \textbf{Responsabilidades} \\
        \hline
        & & \\
        \hline
        & & \\
        \hline
        & & \\
        \hline
    \end{tabularx}
    \caption{Roles y Responsabilidades del Equipo}
\end{table}

\subsection{Tareas y Plazos}
[Definir las tareas específicas del proyecto y establecer plazos para su realización]

\begin{table}[H]
    \centering
    \begin{tabularx}{\textwidth}{|X|X|X|X|X|}
        \hline
        \textbf{Tarea} & \textbf{Responsable} & \textbf{Fecha Inicio} & \textbf{Fecha Fin} & \textbf{Dependencias} \\
        \hline
        & & & & \\
        \hline
        & & & & \\
        \hline
        & & & & \\
        \hline
    \end{tabularx}
    \caption{Cronograma de Tareas}
\end{table}

%------------------------------------------------
% PASO 7: IDENTIFICACIÓN DE KPI
%------------------------------------------------
\section{Paso 7: Identificación de KPI}

\subsection{Definición de KPI}
[Identificar y detallar los Indicadores Clave de Desempeño (KPI) que se incorporarán en el trabajo]

\begin{table}[H]
    \centering
    \begin{tabularx}{\textwidth}{|X|X|X|X|X|}
        \hline
        \textbf{KPI} & \textbf{Descripción} & \textbf{Fórmula de Cálculo} & \textbf{Meta} & \textbf{Impacto en el Proceso} \\
        \hline
        & & & & \\
        \hline
        & & & & \\
        \hline
        & & & & \\
        \hline
    \end{tabularx}
    \caption{Indicadores Clave de Desempeño (KPI)}
\end{table}

\subsection{Justificación de los KPI Seleccionados}
[Explicar por qué se seleccionaron estos KPI específicos y cómo se relacionan con los objetivos del proyecto]

%------------------------------------------------
% PASO 8: DEFINICIÓN Y DESCRIPCIÓN DE LOS DATOS
%------------------------------------------------
\section{Paso 8: Definición y Descripción de los Datos}

\subsection{Establecimiento y Búsqueda del Data Set}
[Describir el proceso de establecimiento y búsqueda del conjunto de datos]

\subsection{Descripción Detallada del Data Set}
[Proporcionar una descripción detallada del conjunto de datos, incluyendo variables, tipos de datos, estadísticas descriptivas, etc.]

\begin{table}[H]
    \centering
    \begin{tabularx}{\textwidth}{|X|X|X|X|}
        \hline
        \textbf{Variable} & \textbf{Tipo de Dato} & \textbf{Descripción} & \textbf{Ejemplo} \\
        \hline
        & & & \\
        \hline
        & & & \\
        \hline
        & & & \\
        \hline
    \end{tabularx}
    \caption{Descripción de Variables del Data Set}
\end{table}

%------------------------------------------------
% PASO 9: PREPARACIÓN DE LOS DATOS
%------------------------------------------------
\section{Paso 9: Preparación de los Datos}

\subsection{Carga de Datos}
[Describir el proceso de carga de los datos]

\subsection{Transformación de Datos}
[Detallar las transformaciones aplicadas a los datos para prepararlos para el análisis]

\begin{table}[H]
    \centering
    \begin{tabularx}{\textwidth}{|X|X|X|}
        \hline
        \textbf{Variable Original} & \textbf{Transformación Aplicada} & \textbf{Variable Resultante} \\
        \hline
        & & \\
        \hline
        & & \\
        \hline
        & & \\
        \hline
    \end{tabularx}
    \caption{Transformaciones Aplicadas a los Datos}
\end{table}

%------------------------------------------------
% PASO 10: ESTABLECIMIENTO DE REGLAS DE CÁLCULO
%------------------------------------------------
\section{Paso 10: Establecimiento de Reglas de Cálculo}

\subsection{Métricas Definidas}
[Describir las métricas definidas para el análisis]

\begin{table}[H]
    \centering
    \begin{tabularx}{\textwidth}{|X|X|X|}
        \hline
        \textbf{Métrica} & \textbf{Fórmula de Cálculo} & \textbf{Descripción} \\
        \hline
        & & \\
        \hline
        & & \\
        \hline
        & & \\
        \hline
    \end{tabularx}
    \caption{Métricas Definidas}
\end{table}

\subsection{Campos Calculados}
[Describir los campos calculados creados para el análisis]

\begin{table}[H]
    \centering
    \begin{tabularx}{\textwidth}{|X|X|X|}
        \hline
        \textbf{Campo Calculado} & \textbf{Fórmula} & \textbf{Descripción} \\
        \hline
        & & \\
        \hline
        & & \\
        \hline
        & & \\
        \hline
    \end{tabularx}
    \caption{Campos Calculados}
\end{table}

%------------------------------------------------
% PASO 11: IMPLEMENTACIÓN DEL DASHBOARD
%------------------------------------------------
\section{Paso 11: Implementación del Dashboard en Power BI}

\subsection{Diseño del Dashboard}
[Describir el diseño general del dashboard y la justificación de este diseño]

\subsection{Elementos Visuales}
[Describir los elementos visuales (gráficos) incorporados en el dashboard]

\begin{table}[H]
    \centering
    \begin{tabularx}{\textwidth}{|X|X|X|}
        \hline
        \textbf{Elemento Visual} & \textbf{Datos Representados} & \textbf{Justificación} \\
        \hline
        & & \\
        \hline
        & & \\
        \hline
        & & \\
        \hline
    \end{tabularx}
    \caption{Elementos Visuales del Dashboard}
\end{table}

\subsection{Capturas de Pantalla del Dashboard}
[Incluir capturas de pantalla del dashboard implementado]

%------------------------------------------------
% PASO 12: CONCLUSIONES Y PROPUESTAS DE MEJORA
%------------------------------------------------
\section{Paso 12: Conclusiones y Propuestas de Mejora}

\subsection{Análisis de los KPI}
[Presentar y analizar los resultados obtenidos para cada KPI]

\subsection{Propuestas de Mejora}
[Establecer propuestas de mejora sobre el proceso intervenido, basadas en los resultados del análisis]

\begin{table}[H]
    \centering
    \begin{tabularx}{\textwidth}{|X|X|X|X|}
        \hline
        \textbf{Propuesta de Mejora} & \textbf{Descripción} & \textbf{Impacto Esperado} & \textbf{Dificultad de Implementación} \\
        \hline
        & & & \\
        \hline
        & & & \\
        \hline
        & & & \\
        \hline
    \end{tabularx}
    \caption{Propuestas de Mejora}
\end{table}

%------------------------------------------------
% CONCLUSIONES GENERALES
%------------------------------------------------
\chapter{Conclusiones Generales}

\section{Resumen del Trabajo Realizado}
[Resumir el trabajo realizado a lo largo del proyecto]

\section{Logros y Resultados Obtenidos}
[Describir los logros y resultados obtenidos con el proyecto]

\section{Dificultades Encontradas}
[Mencionar las dificultades encontradas durante el desarrollo del proyecto y cómo se superaron]

\section{Lecciones Aprendidas}
[Describir las lecciones aprendidas durante el desarrollo del proyecto]

\section{Recomendaciones para Futuros Proyectos}
[Proporcionar recomendaciones para futuros proyectos similares]

%------------------------------------------------
% REFERENCIAS
%------------------------------------------------
\chapter*{Referencias}
\addcontentsline{toc}{chapter}{Referencias}

[Incluir todas las referencias utilizadas en el informe, siguiendo algún estilo bibliográfico estándar como APA o IEEE]

%------------------------------------------------
% ANEXOS
%------------------------------------------------
\chapter*{Anexos}
\addcontentsline{toc}{chapter}{Anexos}

[Incluir cualquier material adicional relevante para el informe, como código, datos adicionales, gráficos complementarios, etc.]

\end{document}