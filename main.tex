% Formato de Informe para Sistemas de Información I
% Universidad Católica del Norte - Coquimbo
% Primer Semestre 2025

\documentclass[12pt,letterpaper]{report}

% Paquetes necesarios
\usepackage[spanish]{babel}
\usepackage[utf8]{inputenc}
\usepackage{graphicx}
\usepackage{fancyhdr}
\usepackage{hyperref}
\usepackage{booktabs}
\usepackage{tabularx}
\usepackage{xcolor}
\usepackage{enumitem}
\usepackage{titlesec}
\usepackage{float}

% Configuración de márgenes
\usepackage[top=2.5cm, bottom=2.5cm, left=3cm, right=3cm]{geometry}

% Configuración de títulos
\titleformat{\chapter}{\normalfont\LARGE\bfseries}{\thechapter.}{1em}{}
\titleformat{\section}{\normalfont\Large\bfseries}{\thesection}{1em}{}
\titleformat{\subsection}{\normalfont\large\bfseries}{\thesubsection}{1em}{}

% Configuración de encabezado y pie de página
\pagestyle{fancy}
\fancyhf{}
\fancyhead[L]{Sistemas de Información I}
\fancyhead[R]{Universidad Católica del Norte}
\fancyfoot[C]{\thepage}
\renewcommand{\headrulewidth}{0.4pt}
\renewcommand{\footrulewidth}{0.4pt}

% Configuración de hipervínculos
\hypersetup{
    colorlinks=true,
    linkcolor=blue,
    filecolor=magenta,      
    urlcolor=cyan,
    pdftitle={Proyecto de Sistemas de Información},
    pdfpagemode=FullScreen,
}

\begin{document}

%------------------------------------------------
% PORTADA
%------------------------------------------------
\begin{titlepage}
    \centering
    \vspace*{1cm}
    \includegraphics[width=0.4\textwidth]{ucn-EIC.pdf}
    \vspace{1cm}
    
    {\LARGE \textbf{UNIVERSIDAD CATÓLICA DEL NORTE}}\\
    \vspace{0.5cm}
    {\large Escuela de Ingeniería}\\
    \vspace{0.3cm}
    {\large ICCI - ITI}\\
    \vspace{0.3cm}
    {\large Coquimbo}\\
    \vspace{1.5cm}
    
    {\Huge \textbf{Proyecto de Sistemas de Información}}\\
    \vspace{0.5cm}
    {\LARGE Sistemas de Información I}\\
    \vspace{1.5cm}
    
    {\large \textbf{Integrantes:}}\\
    \vspace{0.3cm}
    {\large Daniela Castro }
    {\large Vicente Espinoza }\\
    {\large Diego Martínez}\\
    {\large Sebastián Vega}\\
    {\large Gabriel Vergara}\\
    \vspace{1.5cm}
    
    {\large \textbf{Profesor:} Felipe Quiroz}\\
    \vspace{1.5cm}
    
    {\large Primer Semestre 2025}
    
\end{titlepage}

%------------------------------------------------
% ÍNDICE
%------------------------------------------------
\tableofcontents
\newpage

%------------------------------------------------
% INTRODUCCIÓN
%------------------------------------------------
\chapter*{Introducción}
\addcontentsline{toc}{chapter}{Introducción}

La formacion de profesionales es uno de los pilares principales para las universidades, para ello deben pasar un proceso riguroso de varios años para poder salir a un campo laboral a poder aplicar lo aprendido. Sin embargo, no todas las personas que entran a universiades y/o insitutos tienden a completar este proceso de educacion superior, donde el rendimiento de cada estudiante corresponde a una de las cuantas razones validas para que ocurra este suceso. Es por eso que la Universidad Catolica del Norte, comprometida con la excelencia educativa y academica, ha identificado la necesidad de analizar habitos y estilos de vida de los estudiantes de su institucion para determinar como estos impactan en su desempeño universitario.

Durante este proyecto, se busca abordar esta problematica mediante el análisis de un dataset simulado que contiene el registro de 1.000 estudiantes, con mas de 15 variables a analizar las cuales seran detalladas mas adelante. Para ello, se utilizaran tecnicas de analisis de datos, visualizacion en Power BI y KPIs relevantes. Esto permitira tomar decisiones informadas para impulsar y mejorar el desempeño estudiantil.



\newpage

%------------------------------------------------
% DEFINICIÓN "ORGANIZACIÓN E INDUSTRIA"
%------------------------------------------------
\chapter{Definición de la Organización e Industria}

\section{Análisis de la Organización}

\subsection{Misión}
La Universidad Católica del Norte inspirada en los principios del Humanismo Cristiano y la misión de la Iglesia Católica, contribuye a la creación y transferencia del conocimiento, a la formación integral de la persona y el desarrollo tecnológico. Como institución con vocación de servicio y excelencia impulsa desde el Norte de Chile, con las comunidades y el territorio, la sostenibilidad a través de la docencia, investigación y vinculación con el medio

\subsection{Visión}
Desde su identidad católica y vocación de excelencia, ser una universidad referente en su quehacer, que inspirada en el bien común integre disciplinas, tradiciones, culturas y comunidades para transformar vidas y ampliar oportunidades. 

\subsection{Descripción Detallada}
La Universidad Católica del Norte (UCN) fundada en el año 1956, es una institucion privada de educacion superior que reside en Chile, especificamente en las ciudades de Coquimbo y Antofagasta, siendo esta ultima ciudad donde se encuentra la sede principal. Esta universidad forma parte del Consejo de Rectores de las Universidades Chilenas (CRUCH) y de la Red G9, que agrupa a las principales universidades tradicionales no estatales del país. Cuenta  actualmente con mas de 50 carreras de pregrado, las cuales en su mayoria se encuentran en su sede principal, pero tambien ofrece programas de postgrado,magisteres, doctorados y diplomados

\section{Identificación de Stakeholders}
[Identificar y describir los principales grupos de interés (stakeholders) relacionados con la organización y el proyecto]

\begin{table}[H]
    \centering
    \begin{tabularx}{\textwidth}{|X|X|X|}
        \hline
        \textbf{Stakeholder} & \textbf{Rol/Relación} & \textbf{Intereses/Expectativas} \\
        \hline
        & & \\
        \hline
        & & \\
        \hline
        & & \\
        \hline
    \end{tabularx}
    \caption{Identificación de Stakeholders}
\end{table}

\newpage

%------------------------------------------------
% DEFINICIÓN "DATA SET"
%------------------------------------------------
\chapter{Definición del Data Set}

\section{Descripción del Data Set}
[Describir el conjunto de datos que se utilizará en el proyecto, incluyendo su origen, formato, estructura, variables principales, etc.]

\section{Justificación del Data Set}
[Explicar por qué se eligió este conjunto de datos específico y cómo se relaciona con el problema a resolver]

\section{Calidad y Limitaciones de los Datos}
[Analizar la calidad de los datos, identificar posibles problemas (datos faltantes, inconsistencias, etc.) y limitaciones del conjunto de datos]

\newpage

%------------------------------------------------
% PASOS DEL PROYECTO
%------------------------------------------------
\chapter{Desarrollo del Proyecto}

%------------------------------------------------
% PASO 1: ANÁLISIS DE LA ORGANIZACIÓN
%------------------------------------------------
\section{Paso 1: Análisis de la Organización}
\textit{Nota: Algunas de estas informaciones ya fueron presentadas en el capítulo de "Definición de la Organización e Industria". Se puede hacer referencia a dicho capítulo y complementar con información adicional si es necesario.}

\subsection{Factores Organizacionales Centrales}
[Analizar los factores organizacionales centrales mencionados en el documento:
\begin{itemize}
    \item El entorno en que debe funcionar la organización
    \item La estructura de la organización: jerarquía, especialización, rutinas y procesos de negocios
    \item La cultura y las políticas de la organización
    \item El tipo de organización y su estilo de liderazgo
    \item Los principales grupos de interés afectados por el sistema
    \item Los tipos de tareas, decisiones y procesos de negocios que el sistema de información ayudará
\end{itemize}]

%------------------------------------------------
% PASO 2: INTRODUCCIÓN A LA PROBLEMÁTICA
%------------------------------------------------
\section{Paso 2: Introducción a la Problemática}

\subsection{Descripción del Problema}
[Describir detalladamente el problema o la oportunidad de mejora identificada en la organización]

\subsection{Justificación}
[Explicar por qué es importante resolver este problema o implementar esta mejora]

\subsection{Impacto Esperado}
[Describir el impacto esperado de la solución propuesta en la organización]

%------------------------------------------------
% PASO 3: IDENTIFICACIÓN Y DETALLE DEL PROCESO
%------------------------------------------------
\section{Paso 3: Identificación y Detalle del Proceso}

\subsection{Descripción General del Proceso}
[Describir el proceso sobre el cual se va a trabajar]

\subsection{Diagrama del Proceso}
[Incluir un diagrama que represente visualmente el proceso]

\subsection{Detalle de Cada Paso del Proceso}
[Explicar cada uno de los pasos del proceso identificado]

\begin{table}[H]
    \centering
    \begin{tabularx}{\textwidth}{|X|X|X|X|}
        \hline
        \textbf{Paso} & \textbf{Descripción} & \textbf{Responsable} & \textbf{Recursos necesarios} \\
        \hline
        & & & \\
        \hline
        & & & \\
        \hline
        & & & \\
        \hline
    \end{tabularx}
    \caption{Detalle de los Pasos del Proceso}
\end{table}

%------------------------------------------------
% PASO 4: IDENTIFICACIÓN DE PROBLEMA O MEJORA
%------------------------------------------------
\section{Paso 4: Identificación de Problema o Mejora}

\subsection{Árbol de Problemas}
[Incluir y explicar el árbol de problemas que identifica las causas y efectos del problema principal]

\subsection{Matriz de Vester}
[Incluir y explicar la matriz de Vester para la priorización de problemas]

\begin{table}[H]
    \centering
    \begin{tabular}{|c|c|c|c|c|c|c|}
        \hline
        \textbf{Problemas} & \textbf{P1} & \textbf{P2} & \textbf{P3} & \textbf{...} & \textbf{Total Activos} \\
        \hline
        P1 & & & & & \\
        \hline
        P2 & & & & & \\
        \hline
        P3 & & & & & \\
        \hline
        ... & & & & & \\
        \hline
        \textbf{Total Pasivos} & & & & & \\
        \hline
    \end{tabular}
    \caption{Matriz de Vester}
\end{table}

\subsection{Cadena de Valor}
[Incluir y explicar la cadena de valor de la organización, identificando cómo el proceso seleccionado se relaciona con ella]

\subsection{Análisis FODA}
[Incluir y explicar el análisis FODA relacionado con el proceso a intervenir]

\begin{table}[H]
    \centering
    \begin{tabular}{|p{0.45\textwidth}|p{0.45\textwidth}|}
        \hline
        \textbf{Fortalezas} & \textbf{Oportunidades} \\
        \hline
        \begin{itemize}
            \item Fortaleza 1
            \item Fortaleza 2
            \item ...
        \end{itemize} & 
        \begin{itemize}
            \item Oportunidad 1
            \item Oportunidad 2
            \item ...
        \end{itemize} \\
        \hline
        \textbf{Debilidades} & \textbf{Amenazas} \\
        \hline
        \begin{itemize}
            \item Debilidad 1
            \item Debilidad 2
            \item ...
        \end{itemize} & 
        \begin{itemize}
            \item Amenaza 1
            \item Amenaza 2
            \item ...
        \end{itemize} \\
        \hline
    \end{tabular}
    \caption{Análisis FODA}
\end{table}

%------------------------------------------------
% PASO 5: DEFINICIÓN DE OBJETIVOS, ALCANCE Y SOLUCIÓN
%------------------------------------------------
\section{Paso 5: Definición de Objetivos, Alcance y Solución}

\subsection{Objetivos}

\subsubsection{Objetivo General}
[Definir el objetivo general del proyecto]

\subsubsection{Objetivos Específicos}
[Listar y describir los objetivos específicos del proyecto]

\subsection{Alcance}
[Definir el alcance del proyecto, especificando lo que se incluirá y lo que quedará fuera]

\subsection{Descripción de la Solución Propuesta}
[Describir detalladamente la solución propuesta para resolver el problema identificado]

%------------------------------------------------
% PASO 6: PLANIFICACIÓN
%------------------------------------------------
\section{Paso 6: Planificación}

\subsection{Ciclo de Desarrollo}
[Describir el ciclo de desarrollo que se utilizará para implementar la solución]

\subsection{División del Trabajo y Roles del Equipo}
[Especificar cómo se dividirá el trabajo y qué rol desempeñará cada miembro del equipo]

\begin{table}[H]
    \centering
    \begin{tabularx}{\textwidth}{|X|X|X|}
        \hline
        \textbf{Miembro del Equipo} & \textbf{Rol} & \textbf{Responsabilidades} \\
        \hline
        & & \\
        \hline
        & & \\
        \hline
        & & \\
        \hline
    \end{tabularx}
    \caption{Roles y Responsabilidades del Equipo}
\end{table}

\subsection{Tareas y Plazos}
[Definir las tareas específicas del proyecto y establecer plazos para su realización]

\begin{table}[H]
    \centering
    \begin{tabularx}{\textwidth}{|X|X|X|X|X|}
        \hline
        \textbf{Tarea} & \textbf{Responsable} & \textbf{Fecha Inicio} & \textbf{Fecha Fin} & \textbf{Dependencias} \\
        \hline
        & & & & \\
        \hline
        & & & & \\
        \hline
        & & & & \\
        \hline
    \end{tabularx}
    \caption{Cronograma de Tareas}
\end{table}

%------------------------------------------------
% PASO 7: IDENTIFICACIÓN DE KPI
%------------------------------------------------
\section{Paso 7: Identificación de KPI}

\subsection{Definición de KPI}
[Identificar y detallar los Indicadores Clave de Desempeño (KPI) que se incorporarán en el trabajo]

\begin{table}[H]
    \centering
    \begin{tabularx}{\textwidth}{|X|X|X|X|X|}
        \hline
        \textbf{KPI} & \textbf{Descripción} & \textbf{Fórmula de Cálculo} & \textbf{Meta} & \textbf{Impacto en el Proceso} \\
        \hline
        & & & & \\
        \hline
        & & & & \\
        \hline
        & & & & \\
        \hline
    \end{tabularx}
    \caption{Indicadores Clave de Desempeño (KPI)}
\end{table}

\subsection{Justificación de los KPI Seleccionados}
[Explicar por qué se seleccionaron estos KPI específicos y cómo se relacionan con los objetivos del proyecto]

%------------------------------------------------
% PASO 8: DEFINICIÓN Y DESCRIPCIÓN DE LOS DATOS
%------------------------------------------------
\section{Paso 8: Definición y Descripción de los Datos}

\subsection{Establecimiento y Búsqueda del Data Set}
[Describir el proceso de establecimiento y búsqueda del conjunto de datos]

\subsection{Descripción Detallada del Data Set}
[Proporcionar una descripción detallada del conjunto de datos, incluyendo variables, tipos de datos, estadísticas descriptivas, etc.]

\begin{table}[H]
    \centering
    \begin{tabularx}{\textwidth}{|X|X|X|X|}
        \hline
        \textbf{Variable} & \textbf{Tipo de Dato} & \textbf{Descripción} & \textbf{Ejemplo} \\
        \hline
        & & & \\
        \hline
        & & & \\
        \hline
        & & & \\
        \hline
    \end{tabularx}
    \caption{Descripción de Variables del Data Set}
\end{table}

%------------------------------------------------
% PASO 9: PREPARACIÓN DE LOS DATOS
%------------------------------------------------
\section{Paso 9: Preparación de los Datos}

\subsection{Carga de Datos}
[Describir el proceso de carga de los datos]

\subsection{Transformación de Datos}
[Detallar las transformaciones aplicadas a los datos para prepararlos para el análisis]

\begin{table}[H]
    \centering
    \begin{tabularx}{\textwidth}{|X|X|X|}
        \hline
        \textbf{Variable Original} & \textbf{Transformación Aplicada} & \textbf{Variable Resultante} \\
        \hline
        & & \\
        \hline
        & & \\
        \hline
        & & \\
        \hline
    \end{tabularx}
    \caption{Transformaciones Aplicadas a los Datos}
\end{table}

%------------------------------------------------
% PASO 10: ESTABLECIMIENTO DE REGLAS DE CÁLCULO
%------------------------------------------------
\section{Paso 10: Establecimiento de Reglas de Cálculo}

\subsection{Métricas Definidas}
[Describir las métricas definidas para el análisis]

\begin{table}[H]
    \centering
    \begin{tabularx}{\textwidth}{|X|X|X|}
        \hline
        \textbf{Métrica} & \textbf{Fórmula de Cálculo} & \textbf{Descripción} \\
        \hline
        & & \\
        \hline
        & & \\
        \hline
        & & \\
        \hline
    \end{tabularx}
    \caption{Métricas Definidas}
\end{table}

\subsection{Campos Calculados}
[Describir los campos calculados creados para el análisis]

\begin{table}[H]
    \centering
    \begin{tabularx}{\textwidth}{|X|X|X|}
        \hline
        \textbf{Campo Calculado} & \textbf{Fórmula} & \textbf{Descripción} \\
        \hline
        & & \\
        \hline
        & & \\
        \hline
        & & \\
        \hline
    \end{tabularx}
    \caption{Campos Calculados}
\end{table}

%------------------------------------------------
% PASO 11: IMPLEMENTACIÓN DEL DASHBOARD
%------------------------------------------------
\section{Paso 11: Implementación del Dashboard en Power BI}

\subsection{Diseño del Dashboard}
[Describir el diseño general del dashboard y la justificación de este diseño]

\subsection{Elementos Visuales}
[Describir los elementos visuales (gráficos) incorporados en el dashboard]

\begin{table}[H]
    \centering
    \begin{tabularx}{\textwidth}{|X|X|X|}
        \hline
        \textbf{Elemento Visual} & \textbf{Datos Representados} & \textbf{Justificación} \\
        \hline
        & & \\
        \hline
        & & \\
        \hline
        & & \\
        \hline
    \end{tabularx}
    \caption{Elementos Visuales del Dashboard}
\end{table}

\subsection{Capturas de Pantalla del Dashboard}
[Incluir capturas de pantalla del dashboard implementado]

%------------------------------------------------
% PASO 12: CONCLUSIONES Y PROPUESTAS DE MEJORA
%------------------------------------------------
\section{Paso 12: Conclusiones y Propuestas de Mejora}

\subsection{Análisis de los KPI}
[Presentar y analizar los resultados obtenidos para cada KPI]

\subsection{Propuestas de Mejora}
[Establecer propuestas de mejora sobre el proceso intervenido, basadas en los resultados del análisis]

\begin{table}[H]
    \centering
    \begin{tabularx}{\textwidth}{|X|X|X|X|}
        \hline
        \textbf{Propuesta de Mejora} & \textbf{Descripción} & \textbf{Impacto Esperado} & \textbf{Dificultad de Implementación} \\
        \hline
        & & & \\
        \hline
        & & & \\
        \hline
        & & & \\
        \hline
    \end{tabularx}
    \caption{Propuestas de Mejora}
\end{table}

%------------------------------------------------
% CONCLUSIONES GENERALES
%------------------------------------------------
\chapter{Conclusiones Generales}

\section{Resumen del Trabajo Realizado}
[Resumir el trabajo realizado a lo largo del proyecto]

\section{Logros y Resultados Obtenidos}
[Describir los logros y resultados obtenidos con el proyecto]

\section{Dificultades Encontradas}
[Mencionar las dificultades encontradas durante el desarrollo del proyecto y cómo se superaron]

\section{Lecciones Aprendidas}
[Describir las lecciones aprendidas durante el desarrollo del proyecto]

\section{Recomendaciones para Futuros Proyectos}
[Proporcionar recomendaciones para futuros proyectos similares]

%------------------------------------------------
% REFERENCIAS
%------------------------------------------------
\chapter*{Referencias}
\addcontentsline{toc}{chapter}{Referencias}

[Incluir todas las referencias utilizadas en el informe, siguiendo algún estilo bibliográfico estándar como APA o IEEE]

%------------------------------------------------
% ANEXOS
%------------------------------------------------
\chapter*{Anexos}
\addcontentsline{toc}{chapter}{Anexos}

[Incluir cualquier material adicional relevante para el informe, como código, datos adicionales, gráficos complementarios, etc.]

\end{document}